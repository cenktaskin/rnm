\section{Introduction}
In medical procedures(e.g.\ biopsies, drug insertions) medical needles are used as insertion tools into the patients body.
Considering the delicate structure of the surrounding tissue and the following healing process it is imperative to improve this process for increased safety and precision.
Using robots to guide the needle to the desired position is a viable alternative to the manual insertion by a surgeon, and holds possibilities for such improvements. These advancements can be accomplished with modern 3D imaging techniques and trajectory planning.

In this project aim is to develop robotic needle placement system guided by image processing techniques.
Project framework is to use Robotic Operaing System (ROS) in conjunction with the depth camera (Kinect Azure by Microsoft) mounted on the surgical robot arm (Panda by Franka Emika).
In this manner, depth camera will give the necessary visual information to drive the needle to the target position while the kinematics package driving robot to the required positions.

In this document we present a report of the results of this project as a part of the aforementioned course.
The paper is organized in two major sections, namely Kinematics and Vision explaining the implementation in detail, followed by Discussion and Conclusions, among which the former includes the Contributions part where participants explain their share of the workload.